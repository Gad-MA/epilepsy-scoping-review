\section{Introduction}
The estimated global prevalence of epilepsy is 50 million with approximately 5 million new cases diagnosed each year \cite{whoEpilepsy}. Of these, approximately 20–40\% have refractory epilepsy, a version of the disease that is not controlled by typical antiseizure medications \cite{Kwan2000-jq}. Generalized motor seizures and specifically generalized tonic–clonic seizures (GTCS) are considered one of the most dangerous seizure subtypes that are strongly associated with sudden unexpected death in epilepsy (SUDEP) \cite{Devinsky2016-tn}. Effective seizure detection and prediction systems are needed to give caregivers the opportunity to intervene at the proper time and prevent the dangerous consequences of a seizure. Electroencephalography (EEG) is the gold standard method for seizure detection \cite{Noachtar2009-am} and prediction \cite{Rasheed2021-de}, however, it requires the use of either a hat/headset which is deemed uncomfortable and socially stigmatizing by epileptic patients \cite{Hadady2025-gc}. This has driven research toward investigating the feasibility and effectiveness of alternative wearable and non-EEG systems in detecting and predicting seizure fits.

There are several physiological measures can be discriminative of generalized motor seizures besides known motor manifestations. These include electrodermal activity (EDA), muscle activity captured by surface electromyography (sEMG), and cardiovascular or respiratory measures obtained from electrocardiography (ECG) or photoplethysmography (PPG), such as heart rate (HR), heart rate variability (HRV), blood volume pulse (BVP), and oxygen saturation (SpO$_2$) \cite{Casanovas_Ortega2022-yx,Baumgartner2021-fv,Beniczky2016-ra,Baumgartner2019-wy}. It is worth noting that detection of abnormal movement associated with seizures has been performed using accelerometers (ACC), yet motion signals obtained via ACCs exhibit relatively high false alarms when used alone due to other daily activities mimicking seizure-like movements \cite{Atwood2021-ux}. This prompted the shift toward multimodal seizure detection systems, and the incorporation of other either motion-based sensors like gyroscopes (GYR) or the aforementioned physiological sensors.

For seizure prediction, a gold-standard signal has not yet been identified; therefore, progress in this area has been limited. Thus, the autonomic distinctions of the peri-ictal period remain an open research question, and ongoing studies are investigating how features derived from physiological signals can provide reliable predictive markers.

Accordingly, this scoping review aims to: (1) compare different modalities and algorithms used for seizures detection and prediction; (2) identify the most promising sensor combinations and approaches for both purposes; and (3) highlight current research gaps and limitations in this field. The findings of this review are intended to inform and guide future research toward the development of accurate, high-performance systems for the detection and prediction of GTCS.
