\subsection{Participants Demographics}
\subsubsection{Detection}
Nine (34.62\%) studies targeted generalized tonic-clonic and/or tonic-clonic seizures only \cite{Milosevic2016-ee,Wang2022-lt,De_Cooman2018-pq,Poh2012-af,Vakilna2024-hk,Larsen2024-vn,Li2022-ty,Xu2022-tx,Wang2025-my}. Four (15.38\%) studies focused solely on nocturnal seizures \cite{Van_Andel2017-yx,De_Cooman2018-pq,Gheryani2017-yg,Larsen2024-vn}.

Studies included either adults, pediatrics or a mixed population as follows: Five studies \cite{Cogan2017-lg,Hamlin2021-sd,Chowdhury2022-bi,Ali2020-ke,Vakilna2024-hk} focused solely on adults aged 20 \cite{Chowdhury2022-bi} to 64 years \cite{Cogan2017-lg}. From which, only one study provided the mean age \cite{Hamlin2021-sd}, and another the mean and standard deviation \cite{Vakilna2024-hk}. Eleven out of 26 (42.3\%) studies \cite{Yu2023-ss,Milosevic2016-ee,Hegarty-Craver2021-hk,De_Cooman2018-pq,Poh2012-af,Wu2024-yl,Gheryani2017-yg,Ge2023-ab,Wang2025-ql,Jiang2022-zu,Tang2021-td} were conducted exclusively on pediatric patients aged between 1 month \cite{Jiang2022-zu} and 17 years \cite{Hegarty-Craver2021-hk}. Among these, two studies \cite{De_Cooman2018-pq,Milosevic2016-ee} did not specify exact age but indicated inclusion of children, while four provided both mean and median ages only \cite{Yu2023-ss,Poh2012-af,Ge2023-ab,Tang2021-td}. Ten studies \cite{Regalia2019-ch,Nasseri2021-xn,Van_Andel2017-yx,Arends2018-ew,Wang2022-lt,Larsen2024-vn,Dong2022-oo,Li2022-ty,Xu2022-tx,Wang2025-my} included mixed-age populations, with age ranging from 2 \cite{Van_Andel2017-yx} to 75 years \cite{Larsen2024-vn}, including two studies that did not specify the age range \cite{Xu2022-tx,Nasseri2021-xn}. In addition to seizure patients, five studies included control groups to test the performance of their seizure detection methods \cite{Van_Andel2017-yx,Vakilna2024-hk,Wang2022-lt,Larsen2024-vn,Wang2025-my}. 

Sample sizes varied from 7 \cite{De_Cooman2018-pq} to 166 participants \cite{Yu2023-ss} in studies involving children, and from 5 patients \cite{Ali2020-ke} to 36 \cite{Vakilna2024-hk} in studies including adults, and from 4 patients \cite{Li2022-ty} to 135 patients \cite{Regalia2019-ch} in the mixed-age population studies. 

Twenty studies (76.9\%) were conducted in inpatient settings, three studies in outpatient settings \cite{Chowdhury2022-bi,Wang2022-lt,Dong2022-oo}, and three were conducted in both settings \cite{Wang2025-my,Regalia2019-ch,Nasseri2021-xn}. 

\subsubsection{Prediction and Forecasting}
The three prediction studies were more uniform in the demographics of enrolled participants. The number of patients included in data acquisition ranged from 42 \cite{Vieluf2023-ta} to 139 \cite{Vieluf2023-zv}, with mean ages ranged between 9.8 \cite{Meisel2020-ii} and 14 \cite{Vieluf2023-ta} years, indicating a pediatric-dominant cohort. Two studies \cite{Vieluf2023-zv,Vieluf2023-ta} also included a control group (patients with no seizures) to differentiate seizure-related patterns from normal brain activity. All prediction studies were conducted in an inpatient setting. An additional study \cite{Chen2023-ns} involving neonates was excluded as it did not report specific results for motor seizures. Since neonatal seizures are predominantly of focal onset,distinguishing between focal and generalized seizures is unnecessary. Seizures in this period may present with motor features (automatisms, clonic movements, epileptic spasms, myoclonic, or tonic activity), non-motor features (autonomic manifestations or behavioral arrest), or a sequential progression. Typically, neonatal seizure semiology is characterized by focal tonic movements \cite{Ziobro24-neo}.