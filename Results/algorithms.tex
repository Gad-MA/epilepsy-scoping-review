\subsection{Algorithms}

\subsubsection{Deep Learning Methods}
Deep learning models, particularly Convolutional Neural Networks (CNN) and Long Short-Term Memory (LSTM) networks, were used in several studies and showed strong performance. For example, a CNN-LSTM fusion model was the best performer in one study \cite{Yu2023-ss}, and a CNN was used in another \cite{Tang2021-td}. LSTM was also used in \cite{Wang2025-ql} and \cite{Nasseri2021-xn} (with transfer learning).

\subsubsection{Ensemble Methods}
Tree-based ensemble methods like Random Forest and Gradient Boosting have shown good results in seizure detection tasks. For example, Random Forest has been used to detect seizures on a daily basis \cite{Wang2022-lt} and to monitor individual patients \cite{Forooghifar2022-dm}. Gradient Boosting methods, such as XGBoost, worked well for spotting certain types of seizures, like focal to bilateral tonic-clonic seizures \cite{Gharbi2024-ad}, and for tracking children's seizure patterns over time \cite{Jiang2022-zu}. Ensemble methods combining multiple base models further enhanced accuracy. One example is the Two-Layer Ensemble Model (TLEM), which combined four tree-based models (Random Forest, Extra Trees, Gradient Boosting, and AdaBoost) with logistic regression and reached a high sensitivity of 94.57\% for detecting seizures at night \cite{Dong2022-oo}. Bagged decision trees also performed well in real-time seizure detection systems \cite{Chowdhury2022-bi}.

\subsubsection{Traditional Machine Learning}
Traditional machine learning algorithms, especially Support Vector Machines (SVM), were frequently among the best performers. Variants such as Least Squares SVM (LS-SVM) \cite{Milosevic2016-ee}, nonlinear SVM \cite{Onorati2017-bn}, and linear SVM (SVM-L) \cite{Wang2025-my} were used in different contexts, often for multimodal signal analysis.

\subsubsection{Rule-based and Threshold-based Methods}
Rule-based and threshold-based algorithms provided interpretable and efficient solutions, particularly for combining heart rate and motion data. These included stepwise algorithms for nocturnal motor seizures \cite{Van_Andel2017-yx}, multi-parametric threshold-based approaches for cardiac-based detection \cite{Hegarty-Craver2021-hk}, and Shewhart control charts for nocturnal seizures \cite{Gheryani2017-yg}. Fuzzy logic was also employed in an IoT-based monitoring system \cite{Hassan2022-do}.

\subsubsection{Methods with Personalization}
Personalized approaches, such as transfer learning \cite{Nasseri2021-xn} and autoencoders \cite{Yu2023-ss}, were employed to adapt detection models to individual patient physiology, improving sensitivity for specific seizure types.