\subsection{Algorithms}

\subsubsection{Deep Learning Methods}
Studies have leveraged deep learning architectures to automatically learn hierarchical features and model complex temporal dependencies in multimodal sensor data. Five of the reviewed detection papers \cite{Yu2023-ss, Nasseri2021-xn, Larsen2024-vn, Wang2025-ql, Tang2021-td} and two prediction papers \cite{Vieluf2023-ta, Meisel2020-ii} employed deep learning methods. The most prominent architectures included Convolutional Neural Networks (CNNs) \cite{Yu2023-ss, Tang2021-td}, Long Short-Term Memory (LSTM) networks \cite{Meisel2020-ii, Yu2023-ss, Wang2025-ql}, and hybrid models combining both \cite{Yu2023-ss}.

For the seizure detection task, hybrid CNN-LSTM models were shown to be particularly effective. One study identified a CNN-LSTM fusion model using accelerometer (ACC) and blood volume pulse (BVP) data as the best overall algorithm, achieving 83.9\% sensitivity and a detection delay of 28 seconds across 28 seizure types \cite{Yu2023-ss}. For generalized tonic-clonic (GTC) seizures specifically, this model reached 95\% sensitivity \cite{Yu2023-ss}. Standalone LSTM networks were also successfully applied, particularly for their ability to capture time-series dynamics. One such study demonstrated that an LSTM model with transfer learning significantly outperformed traditional learning, achieving 93\% sensitivity for in-hospital motor seizures with a false alarm rate (FAR) of 2.3 per day \cite{Nasseri2021-xn}. Another study utilized an LSTM with attitude angle signals, among others, to achieve an accuracy of 83.4\% \cite{Wang2025-ql}.

Other neural network architectures were also explored. A CNN-based model was found to be feasible for detecting a broad variety of seizure types using ACC and BVP signals \cite{Tang2021-td}. Simpler Artificial Neural Networks (ANNs) also proved effective, with one study reporting 100\% sensitivity in detecting nocturnal tonic seizures in an independent test set \cite{Larsen2024-vn}. In the context of personalization, an autoencoder model achieved 100\% sensitivity for GTC seizures in select patients, although it was noted to be less scalable \cite{Yu2023-ss}.

For the forecasting and prediction tasks, one study \cite{Meisel2020-ii} employed supervised long short-term memory (LSTM) neural networks trained using leave-one-subject-out cross-validation on continuously collected retrospective data. Another study \cite{Vieluf2023-ta} selected Deep Canonically Correlated Autoencoders (DCCAE) for training and validation, testing three different architectures: Fully Connected DCCAE (FC-DCCAE), Convolutional Neural Network DCCAE (CNN-DCCAE), and Gated Recurrent Unit DCCAE (GRU-DCCAE). Among these, GRU-DCCAE yielded the best clustering accuracy of 68.89\%. Both studies lacked real-time implementation, relying instead on retrospective or offline data analysis.


\subsubsection{Ensemble Methods}
Ensemble learning, which combines multiple machine learning models to improve predictive performance, was a prominent and effective strategy in several detection studies. Six detection studies reported the use of ensemble classifiers, including bagging, boosting, and more complex stacked architectures \cite{Wang2022-lt, Chowdhury2022-bi, Vakilna2024-hk, Dong2022-oo, Jiang2022-zu, Wu2024-yl}, and one forecasting study reported the use of random forest classifier \cite{Vieluf2023-zv}.

For the seizure prediction task, bagging-based methods, such as Random Forest and Bagged Decision Trees, were particularly common \cite{Chowdhury2022-bi, Wang2022-lt, Wu2024-yl, Vakilna2024-hk}. One study demonstrated that a Random Forest algorithm achieved 90\% sensitivity with a low false alarm rate of 1.21 per 24 hours for tonic-clonic seizures in a daily setting \cite{Wang2022-lt}. Another study found that a Bagged Decision Tree classifier, when applied to fused accelerometer (ACC) and electrodermal activity (EDA) data, yielded a seizure detection recall of 97.2\% and an overall accuracy of 96.7\% \cite{Chowdhury2022-bi}.

Boosting algorithms, including AdaBoost and XGBoost, were also evaluated, with one study comparing their performance alongside other classifiers like Support Vector Machines (SVM) \cite{Xu2022-tx}. A more advanced approach was taken in a study that developed a Two-Layer Ensemble Method (TLEM), which stacked multiple base learners including Random Forest (RF), Extra Trees (ET), Gradient Boosting Decision Tree (GBDT), and AdaBoost (ADB) \cite{Dong2022-oo}. This stacked model was shown to outperform all of its single-layer components, achieving a particularly high sensitivity of 94.57\% and a false alarm rate of 0.46 per 24 hours for nocturnal seizures \cite{Dong2022-oo}. This body of work indicates that ensemble methods are a robust choice for aggregating information from multimodal sensors and improving the reliability of seizure detection.

For the seizure forecasting task, \cite{Vieluf2023-zv} evaluated the performance of seven supervised learning algorithms with random forest achieving the higher accuracy of 68\% and sensitivity of 62\%.


\subsubsection{Traditional Machine Learning}
Eight detection studies reported the successful application of models such as Support Vector Machines (SVM), K-Nearest Neighbors (KNN), and Linear Discriminant Analysis (LDA) \cite{Milosevic2016-ee, Hamlin2021-sd, Poh2012-af, Ge2023-ab, Li2022-ty, Xu2022-tx, Wang2025-my, De_Cooman2018-pq}.

Support Vector Machines were one of the most frequently and successfully implemented classifiers \cite{Milosevic2016-ee, De_Cooman2018-pq, Poh2012-af, Ge2023-ab, Li2022-ty, Xu2022-tx, Wang2025-my}. One study found that an SVM delivered the best trade-off between accuracy and false alarms, achieving a 100\% accurate recognition rate with just 0.08 false alarms per day \cite{Xu2022-tx}. Another study concluded that a linear SVM (SVM-L) provided the optimal sensitivity and overall performance among several tested models \cite{Wang2025-my}. The effectiveness of SVMs was also noted when using attitude angle signals, where they yielded the highest overall accuracy compared to decision trees and LDA \cite{Wang2025-ql}.

Other traditional classifiers also showed strong performance. KNN, particularly when using a cosine distance metric on features from four combined modalities, achieved a high sensitivity of 88.16\% \cite{Ge2023-ab}. A non-patient-specific classifier achieved 88\% sensitivity with a low false alarm rate of one per 24 hours \cite{Poh2012-af}. The collective results indicate that when paired with robust feature engineering, traditional machine learning models remain a viable and powerful option for developing seizure detection systems.


\subsubsection{Rule-based and Threshold-based Methods}
These methods rely on predefined physiological patterns or thresholds to trigger a seizure detection. Five of the reviewed detection papers used this approach \cite{Cogan2017-lg, Ali2020-ke, Hegarty-Craver2021-hk, Gheryani2017-yg, Arends2018-ew}.

One study developed a system based on a multi-biosignal pattern, defining a seizure event as a sequence of heart rate increase, followed by a decrease in blood oxygen saturation, and a subsequent rise in electrodermal activity \cite{Cogan2017-lg}. This pattern-based method successfully detected all seizures from 6 out of 10 patients in the study \cite{Cogan2017-lg}. Another study implemented a system with decision rules based on three factors: shaking, from an accelerometer, heart rate, and temperature \cite{Ali2020-ke}. By classifying risk into discrete levels, the system achieved a sensitivity of 85\% \cite{Ali2020-ke}. 

A threshold-based algorithm using cardiac features was also reported, detecting 92\% of seizures with tonic/clonic movements \cite{Hegarty-Craver2021-hk}. Another approach applied a Shewhart control chart with exponentially weighted moving averages to inertial and muscular activity, achieving a 97\% detection rate with a 4\% false alarm rate \cite{Gheryani2017-yg}. A further study employed a combined accelerometry and heart rate threshold algorithm in a residential care setting, reaching a median sensitivity of 86\% with a positive predictive value of 49\% \cite{Arends2018-ew}.


\subsubsection{Methods with Personalization}
Recognizing the high degree of inter-patient variability in seizure manifestation, many studies investigated or recommended personalized algorithms. Eight detection papers highlighted the benefits of tailoring models to individual patients \cite{Yu2023-ss, Poh2012-af, Nasseri2021-xn, Milosevic2016-ee, Hamlin2021-sd, Jiang2022-zu, Hegarty-Craver2021-hk, Wang2025-ql}.

One study showed that a "semi-patient-specific" approach, which included prior seizure examples from the test patient in the training data, improved sensitivity from 88\% to 94\% compared to a generic model \cite{Poh2012-af}. Similarly, a personalized autoencoder was able to achieve 100\% sensitivity for GTC seizures in certain individuals, a level of performance not reached by the generalized model \cite{Yu2023-ss}. A particularly effective technique was transfer learning, where a pre-trained general model was fine-tuned on patient-specific data. This method significantly improved performance, reducing the false alarm rate from 11.3 per day with traditional learning to 2.33 per day \cite{Nasseri2021-xn}.

Other studies underscored the need for personalization by observing that false alarms and key predictive features varied significantly between individuals \cite{Milosevic2016-ee, Hamlin2021-sd}. Some methodologies were inherently personalized, such as a system that tracked individual "physiomes" by establishing personal physiological baselines to detect seizure-related deviations \cite{Jiang2022-zu}. The consensus across these studies is that while generalized models can provide a strong out-of-the-box solution, personalization is a critical step for maximizing detection accuracy and minimizing false alarms in real-world applications.
