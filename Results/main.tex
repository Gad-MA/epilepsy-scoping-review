\section{Results}

A total of 26 studies, concerned with seizure detection, were included in this review (Table number). These studies utilized a range of commercial and non-commercial devices for seizure detection. The most frequently used commercial device was the EMPATICA E4, which was tested in 5 studies \cite{Yu2023-ss, Regalia2019-ch,Nasseri2021-xn, Tang2021-td}. Additional studies employed either customized laboratory-developed devices or other commercial devices, such as: Shimmer \cite{Van_Andel2017-yx, Gheryani2017-yg}, Samsung SM-R800 watch \cite{Vakilna2024-hk}, Mi- Microsoft wristband\cite{Jiang2022-zu}, Nightwatch \cite{Arends2018-ew}, and Biovital P1 System \cite{Wu2024-yl, Wang2025-ql}.

All included studies assessed various combinations of the following physiological and movement sensors and measures: Accelerometer (ACC), Blood Volume Pulse (BVP), Gyroscope (Gyro), Electromyography (EMG), Heart rate (HR), Oxygen Saturation (SPO2), Electrodermal activity (EDA), electrocardiography (ECG), temperature, and Heart rate variability (HRV).

For seizure prediction and forecasting, three studies were found eligible for this review (table number. All three papers used the Empatica E4 wristband device for wearable data acquisition, which includes sensors for electrodermal activity (EDA), blood volume pulse (BVP) via photoplethysmography, 3-axis accelerometry (ACC), and skin temperature; however, the selected modalities for training and analysis differed among the papers.

The number of patients included in data acquisition ranged from 42 \cite{Vieluf2023-ta} to 139 \cite{Vieluf2023-zv} with a mean age ranging from 9.8 \cite{Meisel2020-ii} to 14 \cite{Vieluf2023-ta} years old, indicating a pediatric-dominant cohort. Two studies \cite{Vieluf2023-ta, Vieluf2023-zv} included a control group (patients with no seizures) besides the seizure group to differentiate seizure-related patterns from normal brain activity. All studies were conducted in an in-hospital setting.

\subsection{Participants Demographics}
\subsubsection{Detection}
Nine (34.62\%) studies targeted generalized tonic-clonic and/or tonic-clonic seizures only \cite{Milosevic2016-ee,Wang2022-lt,De_Cooman2018-pq,Poh2012-af,Vakilna2024-hk,Larsen2024-vn,Li2022-ty,Xu2022-tx,Wang2025-my}. Four (15.38\%) studies focused solely on nocturnal seizures \cite{Van_Andel2017-yx,De_Cooman2018-pq,Gheryani2017-yg,Larsen2024-vn}.

Studies included either adults, pediatrics or a mixed population as follows: Five studies \cite{Cogan2017-lg,Hamlin2021-sd,Chowdhury2022-bi,Ali2020-ke,Vakilna2024-hk} focused solely on adults aged 20 \cite{Chowdhury2022-bi} to 64 years \cite{Cogan2017-lg}. From which, only one study provided the mean age \cite{Hamlin2021-sd}, and another the mean and standard deviation \cite{Vakilna2024-hk}. Eleven out of 26 (42.3\%) studies \cite{Yu2023-ss,Milosevic2016-ee,Hegarty-Craver2021-hk,De_Cooman2018-pq,Poh2012-af,Wu2024-yl,Gheryani2017-yg,Ge2023-ab,Wang2025-ql,Jiang2022-zu,Tang2021-td} were conducted exclusively on pediatric patients aged between 1 month \cite{Jiang2022-zu} and 17 years \cite{Hegarty-Craver2021-hk}. Among these, two studies \cite{De_Cooman2018-pq,Milosevic2016-ee} did not specify exact age but indicated inclusion of children, while four provided both mean and median ages only \cite{Yu2023-ss,Poh2012-af,Ge2023-ab,Tang2021-td}. Ten studies \cite{Regalia2019-ch,Nasseri2021-xn,Van_Andel2017-yx,Arends2018-ew,Wang2022-lt,Larsen2024-vn,Dong2022-oo,Li2022-ty,Xu2022-tx,Wang2025-my} included mixed-age populations, with age ranging from 2 \cite{Van_Andel2017-yx} to 75 years \cite{Larsen2024-vn}, including two studies that did not specify the age range \cite{Xu2022-tx,Nasseri2021-xn}. In addition to seizure patients, five studies included control groups to test the performance of their seizure detection methods \cite{Van_Andel2017-yx,Vakilna2024-hk,Wang2022-lt,Larsen2024-vn,Wang2025-my}. 

Sample sizes varied from 7 \cite{De_Cooman2018-pq} to 166 participants \cite{Yu2023-ss} in studies involving children, and from 5 patients \cite{Ali2020-ke} to 36 \cite{Vakilna2024-hk} in studies including adults, and from 4 patients \cite{Li2022-ty} to 135 patients \cite{Regalia2019-ch} in the mixed-age population studies. 

Twenty studies (76.9\%) were conducted in inpatient settings, three studies in outpatient settings \cite{Chowdhury2022-bi,Wang2022-lt,Dong2022-oo}, and three were conducted in both settings \cite{Wang2025-my,Regalia2019-ch,Nasseri2021-xn}. 

\subsubsection{Prediction and Forecasting}
The three prediction studies were more uniform in the demographics of enrolled participants. The number of patients included in data acquisition ranged from 42 \cite{Vieluf2023-ta} to 139 \cite{Vieluf2023-zv}, with mean ages ranged between 9.8 \cite{Meisel2020-ii} and 14 \cite{Vieluf2023-ta} years, indicating a pediatric-dominant cohort. Two studies \cite{Vieluf2023-zv,Vieluf2023-ta} also included a control group (patients with no seizures) to differentiate seizure-related patterns from normal brain activity. All prediction studies were conducted in an inpatient setting.

\subsection{Modalities}

% \setlength{\tabcolsep}{12pt} 

\begin{table}
    \caption{Modalities (Detection)}
    \vspace{1em}
    \label{tab:modalities}
    \footnotesize
\begin{tabularx}{\textwidth}{@{}lCCCC@{}}
\toprule
\thead{Modality} & \thead{Sensitivity} & \thead{FAR/24H} & \thead{Accuracy} & \thead{Studies} \\
\midrule
ACC, GYR, sEMG, EDA & [81.69\%--95.24\%] & [0.64--1.21] & [93.16\%--96.81\%] & \cite{Wang2025-ql, Ge2023-ab, Li2022-ty, Wu2024-yl, Wang2022-lt} \\
ACC, PPG & [80\%--86\%] & [0.2609--13.63] & --- & \cite{Ali2020-ke, Tang2021-td, Arends2018-ew, Yu2023-ss} \\
ACC, EDA, PPG & [93\%--100\%] & [0.08--2.339] & --- & \cite{Xu2022-tx, Nasseri2021-xn} \\
ACC, ECG & [87\%--92\%] & --- & --- & \cite{Van_Andel2017-yx, Hegarty-Craver2021-hk} \\
ACC, GYR & [76.84\%--96\%] & [0.98] & [97.28\%] & \cite{Larsen2024-vn, Dong2022-oo} \\
ACC, GYR, sEMG & [97\%--100\%] & --- & --- & \cite{Wang2025-my, Gheryani2017-yg} \\
ACC, sEMG & [90.91\%] & --- & --- & \cite{Milosevic2016-ee} \\
ACC, EDA & [93.9\%--97.2\%] & [0.53--1.8] & [96.7\%] & \cite{Regalia2019-ch, Poh2012-af, Chowdhury2022-bi} \\
ACC, ECG, sEMG & [90.9\%] & --- & --- & \cite{De_Cooman2018-pq} \\
ACC, ECG, EDA, sEMG & --- & --- & --- & \cite{Hamlin2021-sd} \\
ACC, GYR, PPG & [87\%] & [0.21] & [93\%] & \cite{Vakilna2024-hk} \\
ACC, EDA, GYR, PPG & [89\%] & [0.54] & --- & \cite{Jiang2022-zu} \\
EDA, PPG & [100\%] & --- & --- & \cite{Cogan2017-lg} \\
\bottomrule
\end{tabularx}

\end{table}

\begin{figure}
    \centering
    \includegraphics[width=1\textwidth]{Results/figures/percentage_of_studies_using_each_modality.png}
    \caption{Percentage of Studies Using Each Modality. Please note that studies used multiple sensors, however, the combination is not shown in this graph}
    \label{fig:percentage_of_studies_using_each_modality}
\end{figure}

\begin{figure}
    \centering
    \includegraphics[width=1\textwidth]{Results/figures/percentage_of_studies_using_each_biomarker.png}
    \caption{Percentage of Studies Using Each Biomarker}
    \label{fig:percentage_of_studies_using_each_biomarker}
\end{figure}

\begin{figure}
    \centering
    \includegraphics[width=1\textwidth]{Results/figures/freq_of_each_sensor_comp.png}
    \caption{Frequency of Each Sensor Combination}
    \label{fig:freq_of_each_sensor_comp}
\end{figure}

\subsubsection{Detection}
The most frequently used modality was ACC (96.2\%) to capture the convulsive motor activity associated with tonic-clonic seizures. EDA followed as the second most used modality, appearing in half of the studies, while ECG was the least commonly used (15.4\%) (Figure \ref{fig:percentage_of_studies_using_each_modality}).

Several studies used raw sensor signals in addition to extracting specific biomarkers such as HR, HRV, BVP, and SpO2, which were then used as features in seizure detection models. HR was the most commonly used biomarker (34.6\%), while SpO2, audio, Number of Wrist Movements (NOWM),  a derived feature that summarizes hand and wrist movement frequency over time windows, and motion parameters such as PITCH and ROLL, which describe the orientation of the body/limb in the 3D space, were each used in only 3.8\% of the studies  (Figure \ref{fig:percentage_of_studies_using_each_biomarker}).

Altogether, 13 different multimodal sensor combinations were reported (Figure \ref{fig:freq_of_each_sensor_comp}). The most common was ACC + GYR + sEMG + EDA (19.2\%). Almost all studies that directly compared unimodal and multimodal systems \cite{Yu2023-ss,Milosevic2016-ee,De_Cooman2018-pq,Chowdhury2022-bi,Ge2023-ab, Wang2025-my,Tang2021-td,Li2022-ty,Hegarty-Craver2021-hk,Poh2012-af,Hamlin2021-sd,Wu2024-yl} found that multimodal systems outperformed unimodal ones. The only exception was the study by Hegarty-Craver \cite{Hegarty-Craver2021-hk}, where a cardiac algorithm using ECG alone achieved a lower false positive alarm (FPR) (1 per day) compared to ACC + ECG (2 per day). Additionally, for GTCS specifically, ACC alone outperformed other modalities and even multimodal combinations including PPG and EDA in two pediatric studies \cite{Yu2023-ss,Tang2021-td}. 

The most commonly used multimodal sensor combination, ACC + GYR + sEMG + EDA [11, 23, 27, 10, 24] consistently achieved high performance  (accuracy range: 93.16[23]-96.81[10] \%), with the highest sensitivity (95.24\%), accuracy (96.81\%), precision (98.55) and lowest FAR/24h (0.64) reported by Wu et al.  [10]. Among these studies, Wang et al. [11] investigated the use of derived biomarkers (PITCH and ROLL, extracted from ACC and GYR) instead of or in combination with raw ACC and GYR data. They found that substituting ACC with PITCH or ROLL improved performance across all models  (e.g. accuracy improved by approximately 2\%), with the best results obtained using Support Vector Machine (SVM) classifier  (when substituted with PITCH: Accuracy: 95.7\%, Precision: 95.7\%, Recall: 93.8\%; With ROLL: Accuracy: 95.2\%, Precision: 95.7\%, Recall: 92.5\%) compared to the original combination which achieved an accuracy of 93.4\%, precision of 95.8\%, and recall of 90.9\%.

Another commonly used combination was ACC + PPG [18, 4, 9, 1]. Yu et al. [1] found that for generalized motor seizures, ACC + BVP achieved the best performance with a mean AUC-ROC of 0.805, whereas EDA performed worst with a mean AUC-ROC of 0.513. For tonic–clonic seizures specifically, ACC alone yielded the highest performance, with an AUC-ROC of 0.973, a sensitivity of 95\%, and an FPR of 6.2\%. Similarly, Tang et al. [4] reported that ACC alone performed best for tonic–clonic seizures (AUC-ROC of 0.995), while for seizure-type–agnostic classification, the fusion of ACC + BVP achieved superior results, with an AUC-ROC of 0.752.  Another study, Arends et al.  [9], in their in-home nocturnal cohort study, reported that their modality combination sensitivity was significantly high (median 85\%) compared to a rhythmic movement-based bed sensor (median 21\%). They further analyzed feature contributions and showed that HR was the critical modality for true positives (92\%) and also for false positives, while ACC contributed only 8\% of true positives and caused no false alarms.

ACC + EDA was also one of the top and high achieving combinations [2, 22, 17] with a reported highest sensitivity of 97.2\% [17] and the lowest FAR/24h of 0.53 reported [2]. Their results validated Empatica’s multimodal wristbands (E4 and Embrace) as reliable tools for GTCS detection in real-world and Epilepsy Monitoring Units (EMUs) settings. Chowdhury et al. [17] also reported that fusing ACC and EDA significantly improved classification accuracy (96.7\%) and reduced FAR (unspecified) compared to unimodal approaches. Similarly, Poh et al. [22] reported that the overall performance was lower when   only ACC features were included.

While most studies incorporate both physiological and motion-based sensors in their sensor combination modalities, 19.2\% of the studies were purely motion-based, using sensor combinations of ACC + GYR [25, 26], ACC + GYR + sEMG [29, 6] and ACC + sEMG [19]. Among these, the ACC + sEMG + GYR configuration achieved the best performance with a sensitivity range of 97[6]-100[11] \%.    

In addition to studying their significance in seizure detection, a few studies have tested for the optimal sensor placement [19, 21], showing that using sensors on different body locations can reduce FAR and improve performance. Milosevic et al. [19] identified the left wrist (non-dominant hand) and right ankle as optimal positions for ACC sensors, while bilateral biceps were optimal for sEMG.

For details of the remaining reported combinations, refer to the supplementary file (\href{https://docs.google.com/spreadsheets/d/1FjxwkHFbNDM84nuqg513gR_0vIVql-evoT1EMiqSYZU/edit?pli=1&gid=1893827577#gid=1893827577}{S2}).

\subsubsection{Prediction and Forecasting}
All three studies used EDA and PPG in their sensor combinations. One study additionally used ACC [14]. PPG was used to extract biomarkers like HR [12, 13], HRV [13], and BVP [14]. TEMP was also among the biomarkers used [13, 14], however, in one study [13], TEMP did not differentiate between seizure and non-seizure groups and was therefore not included in further analysis. Vieluf et al. [12, 13] identified EDA and HR as containing sufficient seizure-predictive information, with reported performance of 62\% sensitivity, and an accuracy range of 68–68.89\%. These findings were further supported in the study of Vieluf et al. [13] , where HRV shown predictive value with patients with an impending seizure had lower HR and higher HRV compared to seizure-free patients in evening recordings. In the work of Meisel et al. [14],  forecasting performance was highest when all modalities (EDA, BVP, TEMP, ACC) were combined achieving significant seizure forecasting (better-than-chance) in 43\% of patients (30/69), where the mean sensitivity was 75.6\%, the mean time in warning (TiW), the fraction of time spent in warning,  was 47.2\% and the mean prediction horizon was 31.6 minutes. Each modality contributed uniquely, though ACC sometimes reduced performance in worst-performing  patients -defined as those for whom seizure forecasting accuracy wasn’t significantly better than chance.


\subsection{Preprocessing}

\subsubsection{Signal Synchronization and Quality Control}
To match biosignal recordings with reference data like EEG or video, many studies performed time synchronization. This included adjusting for time drift between devices using start and end timestamps or using the network time protocol (NTP) \cite{Onorati2017-bn,Vakilna2024-hk,Tang2021-td}. Some studies checked signal quality and removed invalid parts of the data. For example, segments were excluded if the temperature was too low or too high (less than 27°C or more than 45°C), which indicated the device was not worn properly \cite{Yu2023-ss,Tang2021-td}. Other signals were removed based on quality checks, like low EDA amplitude or poor PPG quality \cite{Ge2023-ab,Arends2018-ew}. Also, the first and last 15 minutes of recordings were sometimes discarded to avoid artifacts from calibration \cite{Yu2023-ss}.

\subsubsection{Noise and Artifact Removal}
Most studies used filtering to reduce unwanted signals caused by movement, other physiological activity, or environmental noise. Bandpass filters were common. For example, accelerometer and gyroscope signals were filtered between 0.2–47 Hz \cite{Milosevic2016-ee,De_Cooman2018-pq}, 1–24 Hz \cite{Wu2024-yl}, or 0.5–35 Hz \cite{Gheryani2017-yg}. EMG signals were filtered between 20–90 Hz \cite{Wu2024-yl} or with a high-pass filter at 20 Hz \cite{Milosevic2016-ee,De_Cooman2018-pq}. ECG signals used a notch filter at 60 Hz to remove electrical interference from power lines \cite{Hamlin2021-sd}. Some studies smoothed signals using moving averages, like a 10-minute average for temperature \cite{Yu2023-ss} or a 15-point average for EDA \cite{Wang2022-lt}. Median filters were also used for EDA signals \cite{Wang2025-ql}. A few studies used wavelet transforms to break down signals like ECG-derived respiration (EDR) to level 7 to improve entropy \cite{Forooghifar2022-dm}, or to clean heart rate and GSR signals \cite{Jiang2022-zu}.

\subsubsection{Data Segmentation and Windowing}
To analyze the signals, the data was divided into short or long windows. The window length depended on the type of seizure being studied. Short windows, between 2 and 10 seconds, were used to detect convulsive seizures. For example, one study used 2-second windows with 75\% overlap for ACC and EMG data \cite{Milosevic2016-ee,De_Cooman2018-pq,Poh2012-af}. Longer windows, from 1 to 7 minutes, were used to detect slower changes in the body, such as with PPG, EDA \cite{Ramirez-Peralta2021-nq}, HRV \cite{Jiang2022-zu}, or for detecting generalized convulsive seizures (GCS) \cite{Vakilna2024-hk}. Some studies removed low-motion periods by checking if acceleration was too low, such as standard deviation below 0.2g \cite{Wang2022-lt,Dong2022-oo}, or only analyzed data when acceleration was above 0.1g \cite{Larsen2024-vn}.

\subsubsection{Class Imbalance Handling}
Since seizures are rare compared to non-seizure events, many studies dealt with this class imbalance. One method was undersampling, where non-seizure data was randomly removed to create a more balanced dataset. For example, one study used a seizure-to-nonseizure ratio of 1:1.5 \cite{Yu2023-ss,Tang2021-td}. Other studies used oversampling, where seizure data was repeated to make it more balanced, especially for short seizures \cite{Larsen2024-vn}. Some used automatic filtering to remove non-seizure data based on certain signal patterns, like when the main frequency of acceleration was below 2 Hz or when the signal had a noncross ratio above 0.9 \cite{Wang2022-lt}.

\subsubsection{Features Extraction}
Most studies created features from the signals to use in machine learning models. Time-domain features included basic statistics like mean, variance, and entropy \cite{Wang2022-lt,Dong2022-oo}, zero-crossing rates \cite{De_Cooman2018-pq}, and counts of local maxima \cite{Milosevic2016-ee}. Frequency-domain features included power in certain frequency bands, like 9–22.5 Hz \cite{Milosevic2016-ee}, FFT peaks \cite{Wang2022-lt}, and heart rate variability from Lomb-Scargle analysis \cite{Ramirez-Peralta2021-nq}. Some studies combined features from different sensors, such as ACC and EDA \cite{Regalia2019-ch,Wu2024-yl}, or used decision-level fusion \cite{Chowdhury2022-bi}. To reduce the number of features and keep only the most useful ones, they used methods like minimum redundancy maximum relevance (mRMR) \cite{Wang2022-lt,Ge2023-ab}, ANOVA \cite{Dong2022-oo}, or the Wilcoxon rank-sum test \cite{Vakilna2024-hk}.

\subsubsection{Normalization and Baseline Correction}
Normalization helped make signals more comparable between different people or recording sessions. Some studies used z-score normalization, which adjusts signals to have a mean of zero and a standard deviation of one \cite{Nasseri2021-xn}. Others used moving baselines, where the recent history of a signal (such as over 60 seconds) was used to calculate thresholds for heart rate or oxygen saturation \cite{Cogan2015-lu}. Personalized baselines were also used, where each subject’s median signal was used as a reference point \cite{Jiang2022-zu}.


\subsection{Algorithms}

\begin{table}
    \footnotesize
    \caption{Algorithms (Detection)}
    \label{tab:algos}
%%%%%%%%%%%%%%%%%%%%%%%%%%%%%%%%%%%%%%%%%%%%%
%% Deep Learning and personalized methods
%%%%%%%%%%%%%%%%%%%%%%%%%%%%%%%%%%%%%%%%%%%%%
\begin{subtable}{\textwidth}

\caption{Deep Learning and Personalized Algorithms}
\label{tab:deep_and_personalized_algos}

\begin{tabularx}{\textwidth}{lCCCcC}
\hline
\thead{Algorithm} & \thead{Studies} & \thead{Real-time\\Analysis} & \thead{Sensitivity} & \thead{FAR} & \thead{AUC-ROC} \\
\hline
\multirow{2}{*}{CNN} & \cite{Yu2023-ss} & \multirow{2}{*}{No} & 95\% & --- & 0.769 \\ 
% \cline{2-2}\cline{4-6}
 & \cite{Tang2021-td} &  & 80\% & 13.63/24h & 0.752 \\ 
\hline
CNN + LSTM & \cite{Yu2023-ss} & No & 83.9\% & --- & 0.789 \\ 
\hline 
LSTM & \cite{Wang2025-ql} & Yes & --- & 8.46/24h & --- \\ 
\hline
ANN & \cite{Larsen2024-vn} & No & 96\% & 0.23/Night & --- \\ 
\hline
Transfer Learning$^a$ & \cite{Nasseri2021-xn} & --- & 67\% & 4.8/24h & 0.97 \\
\hline
Personalized Autoencoder$^a$ & \cite{Yu2023-ss} & No & 100\% & [0.0--95.36]/24h & --- \\
\hline
\end{tabularx}

\vspace{0.5em}

All are inpatient studies.

$^a$Algorithms with personalization.

\vspace{1em}

\end{subtable}

%%%%%%%%%%%%%%%%%%%%%%%%%%%%%%%%%%%%%%%%%%%%%
%% Ensemble Methods
%%%%%%%%%%%%%%%%%%%%%%%%%%%%%%%%%%%%%%%%%%%%%
\begin{subtable}{\textwidth}

\caption{Ensemble Algorithms}
\label{tab:ensemble_algos}

\begin{tabularx}{\textwidth}{lCCCCCC}
\hline
Algorithm & Studies & Inpatient /Outpatient & Real-Time Analysis & Sensitivity & FAR/24h & Accuracy \\
\hline
\multirow{2}{*}{Random Forest} & \cite{Wang2022-lt} & Outpatient & \multirow{2}{*}{No} & 90\% & 1.21 & --- \\
 & \cite{Vakilna2024-hk} & Inpatient &  & 87\% & 0.21 & 93\% \\
\hline
Bagged decision Tree Classifier & \cite{Chowdhury2022-bi} & Outpatient & Yes & --- & --- & 95.1\% \\
\hline
XGBoost & \cite{Jiang2022-zu} & Inpatient & Yes & 80\% & 1.1 & --- \\
\hline
\multirow{2}{*}{Two-Layer Ensemble Model} & \multirow{2}{*}{\cite{Dong2022-oo}} & \multirow{2}{*}{Outpatient} & \multirow{2}{*}{No} & 76.84\% (Overall) & 0.98 (Overall) & 97.28\% (Overall) \\
 &  &  &  & 94.57\% (Overall) & 0.46 (Night) & 91.37\% (Night) \\
\hline
\end{tabularx}

\end{subtable}

\end{table}

\subsubsection{Deep Learning Methods}
Studies have leveraged deep learning architectures to automatically learn hierarchical features and model complex temporal dependencies in multimodal sensor data. Deep learning methods were employed in 19.2\% of the reviewed detection studies \cite{Yu2023-ss, Nasseri2021-xn, Larsen2024-vn, Wang2025-ql, Tang2021-td} and in 66.7\% of the prediction studies \cite{Vieluf2023-ta, Meisel2020-ii}. The most prominent architectures included Convolutional Neural Networks (CNNs) \cite{Yu2023-ss, Tang2021-td}, Long Short-Term Memory (LSTM) networks \cite{Meisel2020-ii, Yu2023-ss, Wang2025-ql}, and hybrid models combining both \cite{Yu2023-ss}.

For the seizure detection task, hybrid CNN-LSTM models were shown to be particularly effective. One study identified a CNN-LSTM fusion model using accelerometer (ACC) and blood volume pulse (BVP) data as the best overall algorithm, achieving 83.9\% sensitivity and a detection delay of 28 seconds across 28 seizure types \cite{Yu2023-ss}. For generalized tonic-clonic (GTC) seizures specifically, this model reached 95\% sensitivity \cite{Yu2023-ss}. Standalone LSTM networks were also successfully applied, particularly for their ability to capture time-series dynamics. One such study demonstrated that an LSTM model with transfer learning significantly outperformed traditional learning, achieving 93\% sensitivity for in-hospital motor seizures with a false alarm rate (FAR) of 2.3 per day \cite{Nasseri2021-xn}. Another study utilized an LSTM with attitude angle signals, among others, to achieve an accuracy of 83.4\% \cite{Wang2025-ql}.

Other neural network architectures were also explored. A CNN-based model was found to be feasible for detecting a broad variety of seizure types using ACC and BVP signals \cite{Tang2021-td}. Simpler Artificial Neural Networks (ANNs) also proved effective, with one study reporting 100\% sensitivity in detecting nocturnal tonic seizures in an independent test set \cite{Larsen2024-vn}. In the context of personalization, an autoencoder model achieved 100\% sensitivity for GTC seizures in select patients, although it was noted to be less scalable \cite{Yu2023-ss}.

For the forecasting and prediction tasks, one study \cite{Meisel2020-ii} employed LSTM neural networks trained using leave-one-subject-out cross-validation on continuously collected retrospective data. Another study \cite{Vieluf2023-ta} selected Deep Canonically Correlated Autoencoders (DCCAE) for training and validation, testing three different architectures: Fully Connected DCCAE (FC-DCCAE), Convolutional Neural Network DCCAE (CNN-DCCAE), and Gated Recurrent Unit DCCAE (GRU-DCCAE). Among these, GRU-DCCAE yielded the best clustering accuracy of 68.89\%. Both studies lacked real-time implementation, relying instead on retrospective or offline data analysis.


\subsubsection{Ensemble Methods}
Ensemble learning, which combines multiple machine learning models to improve predictive performance, was a prominent and effective strategy in several detection studies. Ensemble classifiers were used in 23\% of the detection studies  including bagging, boosting, and more complex stacked architectures \cite{Wang2022-lt, Chowdhury2022-bi, Vakilna2024-hk, Dong2022-oo, Jiang2022-zu, Wu2024-yl}. Also, one forecasting study reported the use of random forest (RF) classifier \cite{Vieluf2023-zv}.

For the seizure detection task, bagging-based methods, such as Random Forest (RF) and Bagged Decision Trees, were particularly common \cite{Chowdhury2022-bi, Wang2022-lt, Wu2024-yl, Vakilna2024-hk}. One study demonstrated that  algorithm achieved 90\% sensitivity with a low false alarm rate of 1.21 per 24 hours for tonic-clonic seizures in a daily setting \cite{Wang2022-lt}. Another study found that a Bagged Decision Tree classifier, when applied to fused accelerometer (ACC) and electrodermal activity (EDA) data, yielded GTCS detection accuracey of 96.7\% \cite{Chowdhury2022-bi}.

A more advanced approach was the Two-Layer Ensemble Method (TLEM), which stacked multiple base learners including RF, Extra Trees (ET), Gradient Boosting Decision Tree (GBDT), and AdaBoost (ADB) \cite{Dong2022-oo}. This stacked model was shown to outperform all of its single-layer components, achieving a particularly high sensitivity of 94.57\% and a FAR of 0.46 per 24 hours for nocturnal seizures. While achieving a sensitivity of 76.84\% and a FAR of 0.98 per 24 hours for overall, day and night, seizures \cite{Dong2022-oo}.

For the seizure forecasting task, Vieluf et al. \cite{Vieluf2023-zv} evaluated the performance of seven supervised learning algorithms with RF achieving the higher accuracy of 68\% and sensitivity of 62\%.


\subsubsection{Traditional Machine Learning}
In detection, 30.8\% of the studies reported the successful application of models such as SVM, K-Nearest Neighbors (KNN), and Linear Discriminant Analysis (LDA) \cite{Milosevic2016-ee, Hamlin2021-sd, Poh2012-af, Ge2023-ab, Li2022-ty, Xu2022-tx, Wang2025-my, De_Cooman2018-pq}.

The most frequently and successfully implemented classifier was SVM \cite{Milosevic2016-ee, De_Cooman2018-pq, Poh2012-af, Ge2023-ab, Li2022-ty, Xu2022-tx, Wang2025-my}. One study found that an SVM delivered the best trade-off between accuracy and false alarms, achieving a 100\% accurate recognition rate with just 0.08 false alarms per day \cite{Xu2022-tx}. Another study concluded that a linear SVM (SVM-L) provided the optimal sensitivity and overall performance among several tested models \cite{Wang2025-my}. The effectiveness of SVMs was also noted when using attitude angle signals, where they yielded the highest overall accuracy compared to decision trees and LDA \cite{Wang2025-ql}.

Other traditional classifiers also showed strong performance. KNN, particularly when using a cosine distance metric on features from four combined modalities, achieved a high sensitivity of 88.16\% \cite{Ge2023-ab}. A non-patient-specific classifier achieved 88\% sensitivity with a low FAR of one per 24 hours \cite{Poh2012-af}.


\subsubsection{Rule-based and Threshold-based Methods}
These methods rely on predefined physiological patterns or thresholds to trigger a seizure detection. This approach was used in 19.2\% of the reviewed detection studies \cite{Cogan2017-lg, Ali2020-ke, Hegarty-Craver2021-hk, Gheryani2017-yg, Arends2018-ew}.

One study developed a system based on a multi-biosignal pattern, defining a seizure event as a sequence of HR increase, followed by a decrease in SpO$_2$, and a subsequent rise in electrodermal activity \cite{Cogan2017-lg}. This pattern-based method successfully detected all seizures from 6 out of 10 patients \cite{Cogan2017-lg}. Another study implemented a system with decision rules based on three factors: shaking, from an ACC, HR, and TEMP \cite{Ali2020-ke}. By classifying risk into discrete levels, the system achieved a sensitivity of 85\% \cite{Ali2020-ke}. 

A threshold-based algorithm using cardiac physiological features was also reported, detecting 92\% of seizures with tonic/clonic movements \cite{Hegarty-Craver2021-hk}. Another approach applied a Shewhart control chart with exponentially weighted moving averages to motion inertial and muscular activity, achieving a 97\% detection rate with a 4\% FAR \cite{Gheryani2017-yg}. A further study employed a combined accelerometry and heart rate threshold algorithm in a residential care setting, reaching a median sensitivity of 86\% with a positive predictive value (PPV) of 49\% \cite{Arends2018-ew}.


\subsubsection{Methods with Personalization}
Recognizing the high degree of inter-patient variability in seizure manifestation, many studies investigated or recommended personalized algorithms. For detection studies, 30.8\% highlighted the benefits of tailoring models to individual patients \cite{Yu2023-ss, Poh2012-af, Nasseri2021-xn, Milosevic2016-ee, Hamlin2021-sd, Jiang2022-zu, Hegarty-Craver2021-hk, Wang2025-ql}.

One study showed that a "semi-patient-specific" approach, which included prior seizure examples from the test patient in the training data, improved sensitivity from 88\% to 94\% compared to a generic model \cite{Poh2012-af}. Similarly, a personalized autoencoder was able to achieve 100\% sensitivity for GTC seizures in certain individuals, a level of performance not reached by the generalized model \cite{Yu2023-ss}. A particularly effective technique was transfer learning, where a pre-trained general model was fine-tuned on patient-specific data. This method significantly improved performance, reducing the FAR from 11.3 per day with traditional learning to 2.33 per day \cite{Nasseri2021-xn}.

Other studies underscored the need for personalization by observing that false alarms and key predictive features varied significantly between individuals \cite{Milosevic2016-ee, Hamlin2021-sd}. Some methodologies were inherently personalized, such as a system that tracked individual "physiomes" by establishing personal physiological baselines to detect seizure-related deviations \cite{Jiang2022-zu}.