\subsection{Modalities}

% \setlength{\tabcolsep}{12pt} 

\begin{table}
    \caption{Modalities (Detection)}
    \vspace{1em}
    \label{tab:modalities}
    \footnotesize
\begin{tabularx}{\textwidth}{@{}lCCCC@{}}
\toprule
\thead{Modality} & \thead{Sensitivity} & \thead{FAR/24H} & \thead{Accuracy} & \thead{Studies} \\
\midrule
ACC, GYR, sEMG, EDA & [81.69\%--95.24\%] & [0.64--1.21] & [93.16\%--96.81\%] & \cite{Wang2025-ql, Ge2023-ab, Li2022-ty, Wu2024-yl, Wang2022-lt} \\
ACC, PPG & [80\%--86\%] & [0.2609--13.63] & --- & \cite{Ali2020-ke, Tang2021-td, Arends2018-ew, Yu2023-ss} \\
ACC, EDA, PPG & [93\%--100\%] & [0.08--2.339] & --- & \cite{Xu2022-tx, Nasseri2021-xn} \\
ACC, ECG & [87\%--92\%] & --- & --- & \cite{Van_Andel2017-yx, Hegarty-Craver2021-hk} \\
ACC, GYR & [76.84\%--96\%] & [0.98] & [97.28\%] & \cite{Larsen2024-vn, Dong2022-oo} \\
ACC, GYR, sEMG & [97\%--100\%] & --- & --- & \cite{Wang2025-my, Gheryani2017-yg} \\
ACC, sEMG & [90.91\%] & --- & --- & \cite{Milosevic2016-ee} \\
ACC, EDA & [93.9\%--97.2\%] & [0.53--1.8] & [96.7\%] & \cite{Regalia2019-ch, Poh2012-af, Chowdhury2022-bi} \\
ACC, ECG, sEMG & [90.9\%] & --- & --- & \cite{De_Cooman2018-pq} \\
ACC, ECG, EDA, sEMG & --- & --- & --- & \cite{Hamlin2021-sd} \\
ACC, GYR, PPG & [87\%] & [0.21] & [93\%] & \cite{Vakilna2024-hk} \\
ACC, EDA, GYR, PPG & [89\%] & [0.54] & --- & \cite{Jiang2022-zu} \\
EDA, PPG & [100\%] & --- & --- & \cite{Cogan2017-lg} \\
\bottomrule
\end{tabularx}

\end{table}

\begin{figure}
    \centering
    \includegraphics[width=1\textwidth]{Results/figures/percentage_of_studies_using_each_modality.png}
    \caption{Percentage of Studies Using Each Modality. Please note that studies used multiple sensors, however, the combination is not shown in this graph}
    \label{fig:percentage_of_studies_using_each_modality}
\end{figure}

\begin{figure}
    \centering
    \includegraphics[width=1\textwidth]{Results/figures/percentage_of_studies_using_each_biomarker.png}
    \caption{Percentage of Studies Using Each Biomarker}
    \label{fig:percentage_of_studies_using_each_biomarker}
\end{figure}

\begin{figure}
    \centering
    \includegraphics[width=1\textwidth]{Results/figures/freq_of_each_sensor_comp.png}
    \caption{Frequency of Each Sensor Combination}
    \label{fig:freq_of_each_sensor_comp}
\end{figure}

\subsubsection{Detection}
The most frequently used modality was ACC (96.2\%) to capture the convulsive motor activity associated with tonic-clonic seizures. EDA followed as the second most used modality, appearing in half of the studies, while ECG was the least commonly used (15.4\%) (Figure \ref{fig:percentage_of_studies_using_each_modality}).

Several studies used raw sensor signals in addition to extracting specific biomarkers such as HR, HRV, BVP, and SpO2, which were then used as features in seizure detection models. HR was the most commonly used biomarker (34.6\%), while SpO2, audio, Number of Wrist Movements (NOWM),  a derived feature that summarizes hand and wrist movement frequency over time windows, and motion parameters such as PITCH and ROLL, which describe the orientation of the body/limb in the 3D space, were each used in only 3.8\% of the studies  (Figure \ref{fig:percentage_of_studies_using_each_biomarker}).

Altogether, 13 different multimodal sensor combinations were reported (Figure \ref{fig:freq_of_each_sensor_comp}). The most common was ACC + GYR + sEMG + EDA (19.2\%). Almost all studies that directly compared unimodal and multimodal systems \cite{Yu2023-ss,Milosevic2016-ee,De_Cooman2018-pq,Chowdhury2022-bi,Ge2023-ab, Wang2025-my,Tang2021-td,Li2022-ty,Hegarty-Craver2021-hk,Poh2012-af,Hamlin2021-sd,Wu2024-yl} found that multimodal systems outperformed unimodal ones. The only exception was the study by Hegarty-Craver \cite{Hegarty-Craver2021-hk}, where a cardiac algorithm using ECG alone achieved a lower false positive alarm (FPR) (1 per day) compared to ACC + ECG (2 per day). Additionally, for GTCS specifically, ACC alone outperformed other modalities and even multimodal combinations including PPG and EDA in two pediatric studies \cite{Yu2023-ss,Tang2021-td}. 

The most commonly used multimodal sensor combination, ACC + GYR + sEMG + EDA [11, 23, 27, 10, 24] consistently achieved high performance  (accuracy range: 93.16[23]-96.81[10] \%), with the highest sensitivity (95.24\%), accuracy (96.81\%), precision (98.55) and lowest FAR/24h (0.64) reported by Wu et al.  [10]. Among these studies, Wang et al. [11] investigated the use of derived biomarkers (PITCH and ROLL, extracted from ACC and GYR) instead of or in combination with raw ACC and GYR data. They found that substituting ACC with PITCH or ROLL improved performance across all models  (e.g. accuracy improved by approximately 2\%), with the best results obtained using Support Vector Machine (SVM) classifier  (when substituted with PITCH: Accuracy: 95.7\%, Precision: 95.7\%, Recall: 93.8\%; With ROLL: Accuracy: 95.2\%, Precision: 95.7\%, Recall: 92.5\%) compared to the original combination which achieved an accuracy of 93.4\%, precision of 95.8\%, and recall of 90.9\%.

Another commonly used combination was ACC + PPG [18, 4, 9, 1]. Yu et al. [1] found that for generalized motor seizures, ACC + BVP achieved the best performance with a mean AUC-ROC of 0.805, whereas EDA performed worst with a mean AUC-ROC of 0.513. For tonic–clonic seizures specifically, ACC alone yielded the highest performance, with an AUC-ROC of 0.973, a sensitivity of 95\%, and an FPR of 6.2\%. Similarly, Tang et al. [4] reported that ACC alone performed best for tonic–clonic seizures (AUC-ROC of 0.995), while for seizure-type–agnostic classification, the fusion of ACC + BVP achieved superior results, with an AUC-ROC of 0.752.  Another study, Arends et al.  [9], in their in-home nocturnal cohort study, reported that their modality combination sensitivity was significantly high (median 85\%) compared to a rhythmic movement-based bed sensor (median 21\%). They further analyzed feature contributions and showed that HR was the critical modality for true positives (92\%) and also for false positives, while ACC contributed only 8\% of true positives and caused no false alarms.

ACC + EDA was also one of the top and high achieving combinations [2, 22, 17] with a reported highest sensitivity of 97.2\% [17] and the lowest FAR/24h of 0.53 reported [2]. Their results validated Empatica’s multimodal wristbands (E4 and Embrace) as reliable tools for GTCS detection in real-world and Epilepsy Monitoring Units (EMUs) settings. Chowdhury et al. [17] also reported that fusing ACC and EDA significantly improved classification accuracy (96.7\%) and reduced FAR (unspecified) compared to unimodal approaches. Similarly, Poh et al. [22] reported that the overall performance was lower when   only ACC features were included.

While most studies incorporate both physiological and motion-based sensors in their sensor combination modalities, 19.2\% of the studies were purely motion-based, using sensor combinations of ACC + GYR [25, 26], ACC + GYR + sEMG [29, 6] and ACC + sEMG [19]. Among these, the ACC + sEMG + GYR configuration achieved the best performance with a sensitivity range of 97[6]-100[11] \%.    

In addition to studying their significance in seizure detection, a few studies have tested for the optimal sensor placement [19, 21], showing that using sensors on different body locations can reduce FAR and improve performance. Milosevic et al. [19] identified the left wrist (non-dominant hand) and right ankle as optimal positions for ACC sensors, while bilateral biceps were optimal for sEMG.

For details of the remaining reported combinations, refer to the supplementary file (\href{https://docs.google.com/spreadsheets/d/1FjxwkHFbNDM84nuqg513gR_0vIVql-evoT1EMiqSYZU/edit?pli=1&gid=1893827577#gid=1893827577}{S2}).

\subsubsection{Prediction and Forecasting}
All three studies used EDA and PPG in their sensor combinations. One study additionally used ACC [14]. PPG was used to extract biomarkers like HR [12, 13], HRV [13], and BVP [14]. TEMP was also among the biomarkers used [13, 14], however, in one study [13], TEMP did not differentiate between seizure and non-seizure groups and was therefore not included in further analysis. Vieluf et al. [12, 13] identified EDA and HR as containing sufficient seizure-predictive information, with reported performance of 62\% sensitivity, and an accuracy range of 68–68.89\%. These findings were further supported in the study of Vieluf et al. [13] , where HRV shown predictive value with patients with an impending seizure had lower HR and higher HRV compared to seizure-free patients in evening recordings. In the work of Meisel et al. [14],  forecasting performance was highest when all modalities (EDA, BVP, TEMP, ACC) were combined achieving significant seizure forecasting (better-than-chance) in 43\% of patients (30/69), where the mean sensitivity was 75.6\%, the mean time in warning (TiW), the fraction of time spent in warning,  was 47.2\% and the mean prediction horizon was 31.6 minutes. Each modality contributed uniquely, though ACC sometimes reduced performance in worst-performing  patients -defined as those for whom seizure forecasting accuracy wasn’t significantly better than chance.
