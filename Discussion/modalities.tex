\subsection{Modalities}

Accelerometry (ACC) was the most commonly used modality and formed the basis of most systems. However, it was frequently combined with other sensors to improve detection performance, particularly for non-motor seizures.
The most common combinations involved motion sensors like accelerometers (ACC) and gyroscopes, which together provided detailed information about movement intensity and orientation. These were often paired with physiological signals such as electromyography (EMG) to detect muscle activation, electrodermal activity (EDA) to monitor changes in skin conductance linked to autonomic responses, and electrocardiography (ECG) to capture heart rate and cardiac irregularities during seizures. By combining external motion data with internal physiological responses, these multimodal systems aimed to improve detection accuracy, reduce false alarms, and enable more reliable recognition of both motor and non-motor seizure events.
Less frequently used modalities, including PPG, heart rate, temperature, and SpO$_2$, appeared mainly as supplementary inputs in broader sensor arrays.
Overall, the trend favored combining motion and physiological signals to increase sensitivity and reliability across different seizure types.
