\subsection{Algorithms}
While algorithmic advances have considerably improved the performance of wearable seizure detection, prediction, and forecasting systems, several caveats and methodological observations merit discussion. Across studies, the algorithmic landscape remains fragmented, with varying levels of maturity and generalizability. Deep learning architectures such as CNNs, LSTMs, and their hybrids demonstrated strong capabilities for modeling temporal and multimodal dependencies. However, their performance was often constrained by limited and homogeneous datasets, typically derived from inpatient environments with small sample sizes. These models’ “black-box” nature also limits clinical interpretability. Although deep learning reduces reliance on hand-crafted feature engineering, its data demands and opacity remain key challenges for practical and ethical deployment in real-world monitoring.

In contrast, ensemble methods like Random Forest, Bagged Trees, and multi-layer ensemble frameworks offered robust and interpretable alternatives. Their ability to integrate multimodal features and resist overfitting made them appealing for heterogeneous datasets, yet their computational load and limited real-time deployment restrict their current clinical utility. Only a few studies implemented ensemble algorithms in live monitoring contexts, suggesting that while accuracy and robustness are achievable offline, further work is needed to optimize latency, battery efficiency, and adaptability to continuous data streams. The gap between algorithmic sophistication and practical implementation remains a key barrier to translation.

Traditional machine learning algorithms, including SVM, KNN, and LDA, continue to perform competitively, particularly in resource-constrained or small-data scenarios. Their interpretability and lower computational demands make them strong candidates for near-term clinical applications. However, their reliance on manually engineered features may limit adaptability across different patient populations and signal conditions. This underscores the importance of standardized preprocessing and feature selection methods to ensure cross-cohort reliability and to reduce the risk of algorithmic bias.

Finally, personalized and hybrid models emerged as a promising direction, leveraging patient-specific training, transfer learning, or adaptive baselines to account for inter-individual physiological variability. These approaches achieved substantial gains in sensitivity and reductions in false alarm rates but raise concerns regarding scalability and privacy. Future algorithmic development may benefit from combining the strengths of multiple paradigms, deep representation learning for feature extraction, ensemble decision fusion for robustness, and personalized adaptation for individual relevance. To bridge the gap between research and clinical deployment, future work should also emphasize explainability, federated learning for decentralized personalization, and real-time validation under ecologically valid conditions.