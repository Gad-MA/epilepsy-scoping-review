\subsection{Algorithms}
While algorithmic advances have improved wearable seizure detection, prediction, and forecasting, key challenges remain. The algorithmic landscape is fragmented, with variable maturity and generalizability. Deep learning models (CNNs, LSTMs, hybrids) effectively capture temporal and multimodal patterns but are limited by small, homogeneous datasets and low interpretability. Their high data demands and “black-box” nature hinder clinical adoption \cite{Kumar2023-yb}.

Ensemble methods (Random Forest, Bagged Trees) offer robustness and interpretability for multimodal data but face computational and real-time deployment challenges. Few studies tested them in live monitoring, indicating a need for optimization in latency, power efficiency, and adaptability.

Traditional ML algorithms (SVM, KNN, LDA) remain competitive for smaller datasets due to their simplicity and interpretability, though they depend on manual feature design and may struggle with cross-patient variability. Standardized preprocessing and feature selection are essential to reduce bias and improve generalizability.

Personalized and hybrid models—using patient-specific training, adaptive baselines, or transfer learning—show strong potential, improving sensitivity and reducing false alarms. However, they raise scalability and privacy concerns. Future research should combine deep representation learning, ensemble decision fusion, and personalized adaptation, while emphasizing explainability, federated learning, and real-world validation.

This review excluded non-original studies, those lacking clinical validation, or addressing non-GTCS seizures (Table S). Some excluded studies, such as one on neonates \cite{Chen2023-ns}, warrant future consideration. Neonatal seizures, often focal with motor or autonomic features \cite{Ziobro24-neo}, differ in presentation and clinical context. Among nonvalidated works, Jahanbekam et al. and Conradsen et al. showed promise with well-documented datasets, though the former relied on synthetic and healthy-volunteer data and lacked real-time testing—suggesting these approaches merit future patient-based validation.