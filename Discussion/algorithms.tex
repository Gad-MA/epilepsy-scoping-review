\subsection{Algorithms}
The reviewed literature demonstrates wide methodological diversity in algorithms for GTCS detection, prediction, and forecasting, ranging from interpretable rule-based systems to advanced deep learning architectures. For detection, Artificial Neural Networks (ANNs) have shown particular promise, with Larsen et al. \cite{Larsen2024-vn} achieving 100\% sensitivity and a false alarm rate of 0.16 per night in nocturnal seizure detection. While CNN and hybrid CNN-LSTM models have been explored, their higher false positive rates remain a barrier to clinical translation \cite{Yu2023-ss, Tang2021-td}. Ensemble methods such as the Multi-Level Dynamic Time Pile (MLDTP) framework \cite{Wu2024-yl} and traditional classifiers like Support Vector Machines (SVMs) \cite{Poh2012-af} also demonstrate strong performance, particularly when validated against real-world non-ictal data. Simpler threshold- and rule-based algorithms provide interpretable, real-time solutions, exemplified by the cardiac-based detector of Hegarty-Craver et al. \cite{Hegarty-Craver2021-hk}, though their limited adaptability reduces applicability across diverse patients.

In forecasting, supervised recurrent neural networks, especially Long Short-Term Memory (LSTM) models, are the dominant approach. Meisel et al. \cite{Meisel2020-ii} reported significant forecasting in 43.5\% of patients without requiring patient-specific data or dependence on seizure type, suggesting broader generalizability than previously assumed. In contrast, Nasseri et al. \cite{Nasseri2021-ny} achieved forecasting in 83\% of focal epilepsy patients, highlighting both the potential of LSTMs and the need to validate generalization across seizure types. Beyond supervised learning, unsupervised methods such as Deep Canonically Correlated Autoencoders (DCCAE) \cite{Vieluf2023-ta} demonstrate the ability to extract predictive features from multimodal physiological signals without labeled seizure events, offering a path forward for wearable-based forecasting where annotation is limited.

A consistent challenge across detection and forecasting is inter-patient variability. While some architectures appear capable of patient-independent generalization, many models fail to provide reliable performance across all individuals. Personalization strategies, such as transfer learning \cite{Nasseri2021-xn} or integration of clinical information \cite{Vieluf2023-zv}, have shown promise in improving model robustness, most notably through reductions in false alarm rates and better adaptation to individual physiological profiles.

Overall, while advanced deep learning and ensemble methods have achieved encouraging results, no single algorithm yet combines the sensitivity, low false alarm rate, generalizability, and real-time capability required for clinical adoption. Future progress will depend on balancing generalizable architectures with personalized adaptation, and on validating algorithms in real-world ambulatory settings to ensure reliability and safety.
