\subsection{Participants Demographics}
\subsubsection{Detection}
While the reviewed studies were conducted on different populations and in different settings, several limitations still limit the generalizability of their results.

For instance, 76.9\% of the reviewed detection studies (including all 11 pediatric studies), were conducted in an inpatient setting. While such settings allow for gold-standard seizure characterization via video EEG, they also restrict patients’ range of movements and daily activities, which limits the applicability of the resulting models to real-world conditions. When tested in outpatient environments, these models often produce relatively higher FARs [24, 17, 3, 2]. However, studies like Poh et al. [22] tried to mitigate their inpatient bias by adding an extensive amount of real-world non-seizure data activity which allowed a more realistic estimation of the FAR. 

Additionally, while there is no known standard for the minimum number of patients for non-EEG seizure detection, a dataset of at least 30 patients is needed so that the results are reproducible. 57.7\% of seizure detection studies had valid data recordings from less than 30 patients [15, 16, 24, 21, 10, 17, 23, 3, 18, 25, 26, 27, 28, 11, 9]. And while adding control groups is recommended for achieving less false alarms, balancing data should be taken into account for better generalizability. For instance, one study [19] had a dataset 56 patients, out of whom, only 7 patients experienced tonic-clonic seizures and a control group of 49 patients. 

This limitation in patient numbers significantly affects the performance of algorithms. For instance, Yu et al. [1] explored 4 different algorithms and the personalized Autoencoder was tested on a small subset limiting its generalizability, even though the total number of patients in the study was 166. Poh et al. [22] experienced a low number of seizures despite having many patients. Additionally, Larsen et al. [25] had a much smaller test set compared to its training set. Overall, these challenges highlight how the size and balance of patient datasets are critical factors in developing reliable seizure detection models.

\subsubsection{Prediction and Forecasting}
The reviewed prediction and forecasting studies were limited and involved only pediatric cohorts in inpatient settings. Given physiological and behavioral differences between pediatric and adult populations, these findings are not generalizable to adults without further validation. Further, despite substantial number of patients (a maximum of 139 patients) included in the datasets, more diverse datasets are necessary to validate and extend these results.