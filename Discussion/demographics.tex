\subsection{Participants Demographics}
\subsubsection{Detection}
Reviewed studies were heterogeneous in populations and settings, limiting generalizability. Notably, 76.9\% (including all pediatric studies) were conducted in an inpatient setting, enabling gold-standard video EEG but restricting movement and real-life activity. Consequently, similar models tested in outpatient environments showed higher FARs \cite{Wang2022-lt,Chowdhury2022-bi,Nasseri2021-xn,Regalia2019-ch}. However, studies like Poh et al. \cite{Poh2012-af} reduced inpatient bias by adding extensive real-world non-seizure activity data, enabling a more realistic FAR estimation.

Additionally, although no standard exists for minimum patient numbers in non-EEG seizure detection, datasets of $\ge$30 patients improve reproducibility. About 57.7\% of studies included fewer than 30 valid patient recordings \cite{Cogan2017-lg,Hamlin2021-sd,Wang2022-lt,De_Cooman2018-pq,Wu2024-yl,Chowdhury2022-bi,Ge2023-ab,Nasseri2021-xn,Ali2020-ke,Larsen2024-vn,Dong2022-oo,Li2022-ty,Xu2022-tx,Wang2025-ql,Arends2018-ew}. Including a control group helps reduce false alarms, and balanced datasets improve generalizability. For example, one study \cite{Milosevic2016-ee} included 56 subjects—7 with tonic-clonic seizures and 49 controls.

This limitation in patient numbers significantly affects the performance of algorithms. For instance, Yu et al. \cite{Yu2023-ss} explored 4 different algorithms and the personalized Autoencoder was tested on a small subset limiting its generalizability, even though the total number of patients in the study was 166. Poh et al. \cite{Poh2012-af} experienced a low number of seizures despite having many patients. Additionally, Larsen et al. \cite{Larsen2024-vn} had a much smaller test set compared to its training set. Overall, these challenges highlight how the size and balance of patient datasets are critical factors in developing reliable seizure detection models.

Limited patient numbers greatly affect algorithm performance. For example, Yu et al. [13] tested a personalized Autoencoder on a small subset despite having 166 patients. Poh et al. [30] had few seizures despite many participants, and Larsen et al. [31] used a much smaller test set than training set. Overall, dataset size and balance are critical for reliable seizure detection models.

An additional study \cite{Chen2023-ns} involving neonates was excluded as it did not report specific results for motor seizures. Since neonatal seizures are predominantly of focal onset,distinguishing between focal and generalized seizures is unnecessary. Seizures in this period may present with motor features (automatisms, clonic movements, epileptic spasms, myoclonic, or tonic activity), non-motor features (autonomic manifestations or behavioral arrest), or a sequential progression. Typically, neonatal seizure semiology is characterized by focal tonic movements \cite{Ziobro24-neo}.

\subsubsection{Prediction and Forecasting}
Reviewed prediction and forecasting studies were limited to pediatric inpatient cohorts. Due to physiological and behavioral differences, their findings cannot be generalized to adults without further validation. Although some included up to 139 patients, more diverse datasets are needed to confirm and extend these results.