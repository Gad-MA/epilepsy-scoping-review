\subsection{Limitations - Prediction and forecasting Only}
Despite the noted progress in the field of seizure prediction or forecasting, several limitations remain. The reviewed studies primarily involved pediatric cohorts. Given physiological and behavioral differences between pediatric and adult populations, these findings may not directly translate to adults without further validation. Larger, more diverse datasets are necessary to validate and extend these results. Additionally, all studies were conducted in inpatient settings, where patients’ daily activities and stress levels differ significantly from outpatient environments. While inpatient monitoring allows for gold-standard seizure characterization via video EEG, the transferability of these models to outpatient settings is limited. It is worth noting that although outpatient data collection still presents challenges such as low data quality and lack of continuous clinical supervision, it is critical for practical seizure forecasting applications.

Another key limitation is that none of the studies performed real-time data analysis. Real-time seizure forecasting is crucial for timely intervention and practical clinical use. Future research should therefore focus on developing algorithms that can process data in real time.

Finally, the optimal timing and duration for data recording in seizure prediction remain unclear. Although a time interval of 9:00 to 9:15 pm was hypothesized to be ideal for predicting seizures  that could occur during nighttime or early morning hours \cite{Vieluf2023-zv}, which are especially important, particularly for pediatric patients, as nighttime supervision can help reduce the risk of sudden unexpected death in epilepsy (SUDEP) \cite{Trivisano2022-zw}, it is necessary to identify the shortest data windows that still enable reliable prediction across different patient groups to improve model accuracy and reliability.

In summary, while the current evidence supports the potential of wearable multimodal sensor data combined with machine learning or deep learning for seizure forecasting, addressing these limitations such as expanding cohorts, validating in outpatient settings, implementing real-time analysis, and optimizing data acquisition will be essential for moving toward clinical application.
