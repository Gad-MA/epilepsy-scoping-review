\subsection{Strengths and Limitations}
This scoping review followed a rigorous methodology to identify and chart relevant studies, with a focus on promising modalities, features, and algorithms that may inform future research.

However, a few limitations exist: Only studies published in the English language were reviewed. Further, studies demographic characteristics were heterogenous and sample size in some studies was small which may limit their generalizability to real world settings. Moreover, in this review, description of study methodology and results were given, however, we did not attempt to assess the quality and robustness of included studies. 

Another key limitation is that only 26.9\% of the detection studies \cite{De_Cooman2018-pq,Chowdhury2022-bi,Ali2020-ke,Wang2025-ql,Hegarty-Craver2021-hk,Arends2018-ew,Jiang2022-zu}, and none of the prediction and forecasting studies performed real-time data analysis highlighting a critical gap between experimental research and practical application. 

Finally, the optimal timing and duration for data recording in seizure prediction remain unclear. Although a time interval of 9:00 to 9:15 pm was hypothesized to be ideal for predicting seizures that could occur during nighttime or early morning hours \cite{Vieluf2023-zv}, which are especially important, particularly for pediatric patients, as nighttime supervision can help reduce the risk of sudden unexpected death in epilepsy (SUDEP) \cite{Trivisano2022-zw}, it is necessary to identify the shortest data windows that still enable reliable prediction across different patient groups to improve model accuracy and reliability
