\subsection{Strengths and Limitations}
This scoping review used a rigorous approach to identify and chart relevant studies, emphasizing promising modalities, features, and algorithms for future research.

However, several limitations exist. Only English-language studies were included, and many had small, heterogeneous samples, limiting generalizability. While methodologies and results were described, study quality and robustness were not formally assessed.

Notably, only 26.9\% of detection studies \cite{De_Cooman2018-pq,Chowdhury2022-bi,Ali2020-ke,Wang2025-ql,Hegarty-Craver2021-hk,Arends2018-ew,Jiang2022-zu} and none of the prediction or forecasting studies performed real-time analysis, revealing a major gap between experimental work and practical deployment.

Finally, the optimal timing and duration for seizure prediction remain unclear. One study \cite{Vieluf2023-zv} suggested 9:00–9:15 p.m. as an ideal window for predicting nighttime or early-morning seizures—critical periods for SUDEP prevention in children \cite{Trivisano2022-zw}. Future research should identify the shortest effective data windows for reliable prediction across diverse populations to enhance model accuracy and clinical utility.
