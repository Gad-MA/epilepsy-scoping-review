\subsection{Preprocessing}
The preprocessing pipeline across the reviewed studies reveals a consistent, multi-stage approach to preparing wearable sensor data for analysis. The initial steps of synchronization and quality control are fundamental for ensuring data integrity, yet there is a notable lack of a standardized protocol. While methods ranging from manual clock alignment \cite{Yu2023-ss} to automated NTP \cite{Vakilna2024-hk} were reported, the reliance on manual or semi-automated processes can be labor-intensive and represents a barrier to the seamless, large-scale deployment of these systems. Future work should focus on developing fully automated and robust algorithms for real-time signal quality assessment and data synchronization.

Following quality control, the application of filtering techniques was nearly universal, with methods appropriately tailored to specific signal modalities, such as band-pass filtering for motion sensors \cite{Wu2024-yl, De_Cooman2018-pq} and high-pass filtering for sEMG \cite{Milosevic2016-ee}. The primary challenge in this stage is the trade-off between noise reduction and the preservation of subtle, seizure-relevant physiological signatures. Similarly, the segmentation of data into windows is a critical design choice that directly influences model performance. The literature showed a clear distinction, with shorter windows (2-10s) used for capturing the rapid dynamics of motor seizures \cite{Milosevic2016-ee, Larsen2024-vn} and longer windows ($>$30s) used for forecasting or analyzing slower autonomic changes \cite{Meisel2020-ii, Jiang2022-zu}. The optimal window size and overlap remain an open question and may require patient-specific or context-aware adaptation.

The pervasive issue of class imbalance was addressed with various strategies, primarily under- and over-sampling \cite{Yu2023-ss, Tang2021-td, Larsen2024-vn}. While effective, simple random undersampling risks discarding valuable information from non-seizure periods, which could be crucial for building robust models that can distinguish seizures from vigorous daily activities. More advanced methods, such as synthetic data generation or cost-sensitive learning algorithms, were not widely reported and represent a significant area for future research. Finally, the use of feature engineering and subsequent selection \cite{Ge2023-ab, Xu2022-tx} versus the end-to-end learning approach of deep learning models marks a key divergence in methodology. While hand-crafted features offer interpretability, the trend towards deep learning suggests a move to reduce reliance on domain-specific feature design. Normalization and baseline correction \cite{Jiang2022-zu, Nasseri2021-xn} were identified as crucial steps for personalization, enabling models to adapt to the significant physiological variability among individuals.

