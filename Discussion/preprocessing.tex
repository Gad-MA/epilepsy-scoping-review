\subsection{Preprocessing}
The reviewed studies followed a multi-stage preprocessing pipeline for wearable sensor data, though lacking standardization. Synchronization and quality control—ranging from manual alignment \cite{Yu2023-ss} to automated NTP \cite{Vakilna2024-hk}—were essential but often manual, limiting scalability. Future work should develop fully automated, real-time synchronization and quality assessment methods.

Filtering was widely applied and modality-specific (e.g., band-pass for motion \cite{Wu2024-yl, De_Cooman2018-pq}, high-pass for sEMG \cite{Milosevic2016-ee}), balancing noise removal with preserving seizure-relevant signals. Window segmentation strongly affected performance—shorter windows (2–10 s) captured rapid motor dynamics \cite{Milosevic2016-ee, Larsen2024-vn}, while longer ones ($>$30 s) suited slower autonomic changes \cite{Meisel2020-ii, Jiang2022-zu}. Optimal windowing likely requires adaptive, patient-specific approaches.

Class imbalance was managed mainly by under-/over-sampling \cite{Yu2023-ss, Tang2021-td, Larsen2024-vn}, though simple undersampling risks losing valuable non-seizure data. Advanced methods like synthetic data generation or cost-sensitive learning remain underused. Feature engineering and selection \cite{Ge2023-ab, Xu2022-tx} offered interpretability, while deep learning favored end-to-end automation. Finally, normalization and baseline correction \cite{Jiang2022-zu, Nasseri2021-xn} were key for personalization, improving adaptation to individual physiological variability.