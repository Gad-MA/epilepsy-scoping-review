\section{Conclusion}
Non-EEG seizure detection and prediction show great potential for improving epilepsy management. The best-performing sensor combination for seizure detection were ACC, GYR, sEMG and EDA , achieving sensitivities up to 95.24\%, accuracies as high as 96.81\%, precision reaching 98.55\%, and low FAR around 0.64 per 24 hours. Using ACC and EDA alone showed strong detection performance with sensitivities 97.2\% and FAR as low as 0.53 per 24 hours, suggesting a simpler, cost-effective alternative. 

For seizure prediction, combinations involving EDA, BVP, ACC, and temperature sensors were most effective, with sensitivities around 75.6\% and successful predictions in 43\% of patients. EDA and HR were identified as containing sufficient seizure-predictive information with performance of 62\% sensitivity. 

Deep learning models, particularly hybrid CNN-LSTM architectures, achieved high sensitivity (~95\%) for detecting GTCS. While ensemble methods like Random Forest and stacking models showed strong accuracy and low FAR, with sensitivities of 94.57\% and FAR around 0.46 for nocturnal seizures. It was noticed that personalization methods such as transfer learning further improved the model's robustness and lowered the FAR. However, further clinical validation should be done in real-time and outpatient settings to be able to advance these systems as practical tools for epilepsy management.

